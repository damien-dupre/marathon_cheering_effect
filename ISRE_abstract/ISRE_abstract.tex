\documentclass[]{elsarticle} %review=doublespace preprint=single 5p=2 column
%%% Begin My package additions %%%%%%%%%%%%%%%%%%%
\usepackage[hyphens]{url}

  \journal{ISRE 2019} % Sets Journal name


\usepackage{lineno} % add
\providecommand{\tightlist}{%
  \setlength{\itemsep}{0pt}\setlength{\parskip}{0pt}}

\bibliographystyle{elsarticle-harv}
\biboptions{sort&compress} % For natbib
\usepackage{graphicx}
\usepackage{booktabs} % book-quality tables
%%%%%%%%%%%%%%%% end my additions to header

\usepackage[T1]{fontenc}
\usepackage{lmodern}
\usepackage{amssymb,amsmath}
\usepackage{ifxetex,ifluatex}
\usepackage{fixltx2e} % provides \textsubscript
% use upquote if available, for straight quotes in verbatim environments
\IfFileExists{upquote.sty}{\usepackage{upquote}}{}
\ifnum 0\ifxetex 1\fi\ifluatex 1\fi=0 % if pdftex
  \usepackage[utf8]{inputenc}
\else % if luatex or xelatex
  \usepackage{fontspec}
  \ifxetex
    \usepackage{xltxtra,xunicode}
  \fi
  \defaultfontfeatures{Mapping=tex-text,Scale=MatchLowercase}
  \newcommand{\euro}{€}
\fi
% use microtype if available
\IfFileExists{microtype.sty}{\usepackage{microtype}}{}
\ifxetex
  \usepackage[setpagesize=false, % page size defined by xetex
              unicode=false, % unicode breaks when used with xetex
              xetex]{hyperref}
\else
  \usepackage[unicode=true]{hyperref}
\fi
\hypersetup{breaklinks=true,
            bookmarks=true,
            pdfauthor={},
            pdftitle={You'll never run alone. Effect of cheering zones on athlete performance in marathon races.},
            colorlinks=false,
            urlcolor=blue,
            linkcolor=magenta,
            pdfborder={0 0 0}}
\urlstyle{same}  % don't use monospace font for urls

\setcounter{secnumdepth}{0}
% Pandoc toggle for numbering sections (defaults to be off)
\setcounter{secnumdepth}{0}
% Pandoc header



\begin{document}
\begin{frontmatter}

  \title{You'll never run alone. Effect of `cheering zones' on athlete
performance in marathon races.}
    \author[University College Dublin]{Damien Dupré\corref{c1}}
   \ead{damien.dupre@ucd.ie} 
   \cortext[c1]{Corresponding Author}
    \author[University College Dublin]{Aonghus Lawlor}
   \ead{aonghus.lawlor@ucd.ie} 
  
    \author[University College Dublin]{Barry Smyth}
   \ead{barry.smyth@ucd.ie} 
  
      \address[University College Dublin]{The Insight Centre for Data Analytics, Belfield, Dublin 4}
  
  \begin{abstract}
  
  \end{abstract}
  
 \end{frontmatter}

Although the emotion literature advocates for an influence of positive
emotions on sports performance (McCarthy 2011; Vast, Young, and Thomas
2010), quantifying this influence remains a challenge. Among remarkable
athletic performances, marathon races are a relevant example of this
influence. The establishment of ``cheering zones'' during marathon races
shows how positive emotions and social support are important for
athletes to enhance their performances (Buman et al. 2008). Our aim is
to quantify the behavioural impact of these cheering zones on athletes
pace during marathon races.

In collaboration with Strava Inc. (athlete monitoring application) we
analysed the data of 664 athletes who have finished the Dublin marathon
in 2014. By analyzing their GPS information the Strava app is
calculating the evolution of athletes' pace (min/km) during the
marathon. We compared athletes' pace before, during and after the
cheering zones in order to identify the influence of positive emotion
and social support on athletes' performance.

Generalized Linear Models show not only an effect of cheering zones on
atheltes' pace (\emph{t} = -2.79, \emph{p} \textless{}0.005) but also an
effect of the localisation of these cheering zones (\emph{t} = -3.84,
\emph{p} \textless{}0.001). Athletes tend to increase their pace by
0.743\% after each cheering zones on average but this effect tend to
decrease along the marathon race. This last result is supported by the
comparison athlete's pace comparison before and after the cheering zones
which is significant only by taken into account their localisation
(\emph{t} = -35.18, \emph{p} \textless{}0.001).

Our results are supporting the theory of individual zones of optimal
functioning (IZOF) for which feeling the support of others in cheering
zones would helps athletes to find the motivation to sublim their
performance (Hagtvet and Hanin 2007).

\section*{References}\label{references}
\addcontentsline{toc}{section}{References}

\hypertarget{refs}{}
\hypertarget{ref-buman2008experiences}{}
Buman, Matthew P, Jens W Omli, Peter R Giacobbi Jr, and Britton W
Brewer. 2008. ``Experiences and Coping Responses of `Hitting the Wall'
for Recreational Marathon Runners.'' \emph{Journal of Applied Sport
Psychology} 20 (3). Taylor \& Francis: 282--300.

\hypertarget{ref-hagtvet2007consistency}{}
Hagtvet, Knut A, and Yuri L Hanin. 2007. ``Consistency of
Performance-Related Emotions in Elite Athletes: Generalizability Theory
Applied to the Izof Model.'' \emph{Psychology of Sport and Exercise} 8
(1). Elsevier: 47--72.

\hypertarget{ref-mccarthy2011positive}{}
McCarthy, Paul J. 2011. ``Positive Emotion in Sport Performance: Current
Status and Future Directions.'' \emph{International Review of Sport and
Exercise Psychology} 4 (1). Taylor \& Francis: 50--69.

\hypertarget{ref-vast2010emotions}{}
Vast, Robyn Louise, Robyn Louise Young, and Patrick Robert Thomas. 2010.
``Emotions in Sport: Perceived Effects on Attention, Concentration, and
Performance.'' \emph{Australian Psychologist} 45 (2). Wiley Online
Library: 132--40.

\end{document}


